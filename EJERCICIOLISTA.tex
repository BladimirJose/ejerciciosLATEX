\documentclass[a4paper,11pt]{article}
\usepackage[utf8]{inputenc}
\usepackage[spanish]{babel}
\usepackage{graphicx}
\usepackage{amsmath}
\usepackage{amsfonts}
\usepackage{amssymb}
\begin{document}
\begin{center}
 PRINCIPALES PAÍSES DE AMÉRICA LATINA 
\end{center}

\begin{enumerate}

\item MÉXICO
 \item BRASIL
 \item COLOMBIA 
 \item ARGENTINA
 \item ECUADOR
 \item CHILE
 \item URUGUAY
 \item PARAGUAY
 \item BOLIVIA
 \item PERÚ
 
\end{enumerate}
\begin{flushleft}

A continuación se representa gráficamente la participación del sector Transporte en la producción agregada nacional, para cada uno de los países de América Latina 

\end{flushleft}
 

\begin{figure}[h]
\graphicspath{ {images/} }
\begin{center}
\includegraphics[width=100mm]{PIB.png}
\end{center}
\caption{impacto del transporte en el PIB}
\label{fig:1}
\end{figure}

Ahora procederemos a mostrar a través de una ecuación como se relaciona el cambio de una variable X con respecto a otra Y 

\begin{displaymath}
\frac{\displaystyle
\sum_{i=1}^n(x_i-\overline x)(y_i-\overline y)}
{\displaystyle\biggl[\sum_{i=1}^n(x_i-\overline x)^2\sum_{i=1}^n(y_i-\overline y)^2
\biggr]^{1/2}}
\end{displaymath}
\end{document}
